\documentclass[10pt,twocolumn,letterpaper]{article}

\usepackage{cvpr}
\usepackage{times}
\usepackage{epsfig}
\usepackage{graphicx}
\usepackage{amsmath}
\usepackage{amssymb}
\usepackage[breaklinks=true,bookmarks=false]{hyperref}
\usepackage{color}
\usepackage{indentfirst}
\usepackage{listings}
\definecolor{codegreen}{rgb}{0,0.6,0}
\definecolor{codegray}{rgb}{0.5,0.5,0.5}
\definecolor{codepurple}{rgb}{0.58,0,0.82}
\definecolor{backcolour}{rgb}{0.95,0.95,0.92}


\lstdefinestyle{mystyle}{
  backgroundcolor=\color{backcolour},
  commentstyle=\color{codegreen},
  keywordstyle=\color{magenta},
  numberstyle=\tiny\color{codegray},
  stringstyle=\color{codepurple},
  basicstyle=\footnotesize,
  breakatwhitespace=false,
  breaklines=true,
  captionpos=b,
  keepspaces=true,
  numbers=left,
  numbersep=5pt,
  showspaces=false,
  showstringspaces=false,
  showtabs=false,
  tabsize=2
}
\lstset{style=mystyle}
\cvprfinalcopy % *** Uncomment this line for the final submission

\def\cvprPaperID{****} % *** Enter the CVPR Paper ID here
\def\httilde{\mbox{\tt\raisebox{-.5ex}{\symbol{126}}}}

% Pages are numbered in submission mode, and unnumbered in camera-ready
%\ifcvprfinal\pagestyle{empty}\fi
\setcounter{page}{4321}
\begin{document}

%%%%%%%%% TITLE
\title{
Computer Vision Course Project Report\\ 
Gambody
}

\author{Rong Yuyang, Cai Jianxion, Li Ziyue, Huang Jingyi, Pang Anqi\\
School of Information Science and Technolody\\
ShanghaiTech University\\
{\tt\small \{rongyy, caijx, lizy, huangjy, pangaq\}@shanghaitech.edu.cn}
}

\maketitle

\begin{abstract}

\end{abstract}
\section{Introduction}
\section{Related Work}
\section{Our Approach}
	\subsection{Mask Generation}
		\par We generate mask by taking a photo of a pose and the background first. 
		Then we substract the image to get the binary mask of the pose. 
		We also use Openpose to generate a skeleton data and image.
		\begin{figure}[h]
			\centering
			\includegraphics[width=0.45\linewidth]{./Pic/Approach_Mask_back}
			\includegraphics[width=0.45\linewidth]{./Pic/Approach_Mask_pose}
			\caption{Initial images given}
		\end{figure}
		\begin{figure}[h]
			\centering
			\includegraphics[width=0.9\linewidth]{./Pic/Approach_Mask_binary_mask}
			\includegraphics[width=0.9\linewidth]{./Pic/Approach_Mask_skeleton}
			\caption{Mask generated}
		\end{figure}
	\subsection{Version One}
	\subsection{Openpose\cite{cao2017realtime}}
		\par To generate a skeleton we have to generate confidence map using which we can determine where the joint is. 
		Each joint is self-defined, so joint here can be eyes and ears. 
		18 joints is defined, which is eyes(2), ears(2), nose(1), chest(1), shoulders(2), elbows(2), wrists(2), waists(2), knees(2) and ankles(2).
		We also have to predict Part Affine Fields, which can be used to generate a cost for each connection.
		\par With joints and connection cost, how to connect joints become a bipartite graph maximum matching problem.
		\par We used the binary executable\cite{cao2017realtime} when testing. We will take the last image and give it to the executable. The executable will generate a json file describing each detected joint for each person. 
		\subsubsection{Json}
			
	\subsection{Noise}
\section{Experiment}
	\subsection{Weight on Skeleton}
	\subsection{Noise Cancellation Parameter}
\section{Conclusion}


{\small
\bibliographystyle{ieee}
\bibliography{report}
}
\end{document}
